\documentclass[16pt]{article}
\usepackage[russian]{babel}

\usepackage{amsmath}
\usepackage{graphicx}
\usepackage[colorlinks=true, allcolors=blue]{hyperref}

\title{Нетипизированное $\lambda$ исчисление}
\author{Бабин Руслан}
\date{}

\begin{document}
\maketitle

\section{Привести к нормальной форме $\lambda$ -терм}
Исходный $\lambda$ - терм:
\begin{equation*}
    ((\lambda a.\:(\lambda b.b\;b)\:(\lambda b. b\;b) b))\:((\lambda c.(c\;b))\:(\lambda a.a)) 
\end{equation*}
Применим нормальную стратегию:
\begin{align*}
    & ((\lambda a.\:(\lambda b.b\;b)\:(\lambda b. b\;b) b))\:((\lambda c.(c\;b))\:(\lambda a.a)) \to_\beta \\ 
    & ((\lambda a.\:(\lambda b.b\;b)\:(\lambda b. b\;b)[a := b]))\:((\lambda c.(c\;b))\:(\lambda a.a)) = \\
    & ((\lambda b.b\;b)\:(\lambda b. b\;b))\:((\lambda c.(c\;b))\:(\lambda a.a)) \to_\beta \\
    & ((\lambda b.b\;b)[b := \lambda b.b\;b])\:((\lambda c.(c\;b))\:(\lambda a.a)) = \\
    & ((\lambda b.b\;b)\:(\lambda b. b\;b))\:((\lambda c.(c\;b))\:(\lambda a.a)) 
\end{align*}
Из теоремы Карри о нормализации следует, что нормальной формы у данного $\lambda$-терма нет

\section{Доказать, что $S\ K\ K = I$}
\begin{align*}
    & S\ K\ K \equiv \\
    & (\lambda x y z.x\;z\;(y\;z))\;K\;K \to_\beta \\
    & (\lambda y z.K\;z\;(y \;z)) \ K \equiv \\
    & (\lambda y z.(\lambda x y.x)\;z\;((\lambda x y.x))\;z) \to_\beta \\
    & (\lambda y z.(\lambda x y.x)[x := z]\;(\lambda x y.x)[x := z]) = \\
    & (\lambda y z.(\lambda y.z)(\lambda y.z)) \to_\beta \\
    & (\lambda y z.(\lambda y.z)[y = \lambda y.z]) = \\
    & (\lambda y z.\lambda y.z) \to_\beta \\
    & (\lambda y z)[y := \lambda y.z] = \\
    & \lambda z.z \equiv \\
    & I
\end{align*}

\end{document}
